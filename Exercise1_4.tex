\documentclass{article}

\begin{document}
Suppose learning updates occurred after all moves, including 
exploratory moves. If the step-size parameter is appropriately 
reduced over time, then the state values would converge to a set of 
probabilities. What are the two sets of probabilities computed 
when we do, and when we do not, learn from exploratory moves? 
Assuming that we do continue to make exploratory moves, which set 
of probabilities might be better to learn? Which would result in 
more wins?

\textsc{Solution:} If learning updates occurred after all moves, 
including exploratory moves, and the step-size parameter is 
appropriately reduced over time, the state values would converge to 
a set of probabilities. These probabilities would represent the 
likelihood of winning the game from each possible board state.

When we do not learn from exploratory moves, the 
probabilities would be biased towards the moves that are most 
likely to lead to a win, based on the current learned values. This 
would result in a more exploitative playing style that 
favors safe moves and avoids risks.

On the other hand, if we do learn from exploratory moves, the 
probabilities would be more evenly distributed and would reflect the 
algorithm's exploration of different strategies. This would 
result in a more exploratory playing style that takes risks and 
tries different strategies.

Assuming that we do continue to make exploratory moves, it would 
likely be better to learn from them, as this would allow the 
algorithm to explore different strategies and potentially 
discover better ones. While an exploitative playing style may 
result in some wins, it is more likely to get stuck in suboptimal 
moves and miss out on better strategies.

Therefore, learning from exploratory moves would likely 
result in more wins overall, as it would allow the algorithm to 
discover better strategies and make more informed decisions. 
However, it is important to note that the step-size parameter must 
be appropriately reduced over time to ensure that the state values 
converge to the correct probabilities and that the 
algorithm does not get stuck in a local optimum.
\end{document}
