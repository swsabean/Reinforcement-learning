\documentclass{article}

\begin{document}
1.2 Many tic-tac-toe positions appear different but are really 
the same because of symmetries. How might we amend the reinforcement learning 
algorithm described above to take advantage of this? In what ways would this improve it? Now think 
again. Suppose the opponent did not take advantage of symmetries. In that case, should we? Is it 
true. then. that symmetrically equivalent positions 
should necessarily have the same value?

\textsc{Solution:} To take advantage of symmetries in tic-tac-toe, 
we could modify the reinforcement learning algorithm described above 
to use a symmetric board representation. Instead of 
treating each board state as a unique position, we could consider
symmetrically equivalent positions as the same position. For example,
if a board state is rotationally symmetric, we could treat all 
rotations of that state as the same position. This would reduce 
the number of unique positions that the algorithm needs to learn 
and would make the learning process more efficient.

To implement this modification, we could define a function that takes 
a board state and returns all of its symmetric equivalents. Then, 
we would update the values for all of these equivalent states 
together when the algorithm learns from a particular state. This would ensure that the algorithm 
learns the same value for all equivalent positions, regardless of which position it encounters 
during gameplay.

If the opponent did not take advantage of symmetries, we could
still use the modified algorithm to take advantage of them. This would likely improve the 
algorithm's performance by reducing the number of unique 
positions that it needs to learn and increasing the efficiency of 
the learning process. However, it is not necessarily true that 
symmetrically equivalent positions should always have the same value. In some cases, the optimal move 
for a given symmetric position may depend on the specific position of
the opponent's pieces, which could result in different values for 
symmetrically equivalent positions.
\end{document}
