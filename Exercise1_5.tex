\documentclass{article}

\begin{document}
1.5 Can you think of other ways to improve the reinforcement learning 
player? Can you think of any better way to solve the 
tic-tac- toe problem as posed?

\textsc{Solution:} There are several ways to improve the 
reinforcement learning player in tic-tac-toe, some of which are:
\begin{enumerate}
\item Adding more features: The current algorithm only considers 
the board state when making a move. Adding additional features, 
such as the number of X's and O's in each row, column, and diagonal, 
or the number of potential winning lines for each player, could 
improve the algorithm's ability to evaluate board states and make 
better decisions.

\item Using more advanced techniques: The current algorithm 
uses simple tabular Q-learning. More advanced techniques, such as 
deep reinforcement learning or Monte Carlo tree search, could 
potentially improve the algorithm's performance by 
allowing it to learn more complex strategies and explore a wider 
range of moves.

\item Using human expert knowledge: Incorporating human 
expert knowledge into the algorithm could improve its 
performance by providing it with additional insights and 
strategies. For example, a human expert could provide a set of 
heuristics that the algorithm could use to guide its 
decision-making.
\end{enumerate}

Regarding a better way to solve the tic-tac-toe problem, it is 
important to note that tic-tac-toe is a relatively simple game with a 
small search space, and it is possible to solve it using various 
methods. One approach that could potentially solve tic-tac-toe more 
efficiently than reinforcement learning is the minimax algorithm 
with alpha-beta pruning. This algorithm systematically explores 
all possible moves and uses heuristics to evaluate the value 
of each move. With alpha-beta pruning, it is possible to 
eliminate search branches that are guaranteed to lead to worse 
outcomes, which can significantly reduce the search space and 
improve performance. However, this algorithm requires a good 
evaluation function and is less flexible than reinforcement 
learning, as it cannot learn from experience and adapt to new 
strategies.
\end{document}
