\documentclass{article}

\begin{document}
Suppose the reinforcement learning player was greedy, 
that is, it always played the move that brought it to the position 
that it rated the best. Would it learn to play better, or worse, 
than a nongreedy player? What problems might occur?

\textsc{Solution:} If the reinforcement learning player was 
always greedy and chose the move that brought it to the position it
rated the best, it would likely learn to play worse than a 
non-greedy player. The reason for this is that the greedy player 
would only consider immediate rewards and would not explore 
alternative moves that could lead to greater rewards in the long run.

A non-greedy player, on the other hand, would explore different 
moves and take risks that could lead to greater rewards in the 
long term. This would allow the non-greedy player to discover 
better strategies and avoid getting stuck in suboptimal moves 
that only provide short-term rewards.

The problem with the greedy approach is that it may result in 
the reinforcement learning player getting stuck in a local optimum 
and missing out on a better global optimum. For example, the greedy 
player may choose a move that provides an immediate reward, but 
that move could lead to a worse position in the future that 
reduces the overall reward. By exploring alternative moves, a 
non-greedy player could discover a better strategy that provides 
greater rewards in the long term.

In summary, while the greedy approach may provide some short-
term gains, it is likely to lead to worse overall performance in 
the long term due to the player's lack of exploration and 
risk-taking.
\end{document}
